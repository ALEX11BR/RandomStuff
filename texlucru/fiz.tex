\documentclass{article}
\usepackage{enumitem}
\usepackage{indentfirst}
\usepackage{listings}
\usepackage{graphicx}
\usepackage{caption}
\usepackage[a4paper,portrait,margin=1in]{geometry}
\usepackage{multicol}
\pagenumbering{gobble}

\title{Efectul fotoelectric extern}
\date{}
\begin{document}
\maketitle

\section*{Definiție}
Fenomenul de emisie de electroni de către un corp aflat sub acțiunea radiațiilor electromagnetice.
\section*{Legile efectului fotoelectric}
\begin{itemize}
	\item Intensitatea curentului fotoelectric de saturație este direct proporțională cu fluxul radiațiilor electromagnetice incidente când frecvența e constantă.
	\item Energia cinetică a fotoelectronilor crește liniar cu frecvența radiațiilor electromagnetice și nu depinde de fluxul acestora.
	\item Efectul fotoelectric extern se poate produce numai dacă frecvența radiațiilor precedente este $\geq$ cu o valoare minimă specifică fiecărei substanțe.
	\item Efectul fotoelectric se produce instantaneu.
\end{itemize}

\section*{Ecuații}
\begin{multicols}{2}
$$
h\nu = L + \frac{mv^2}{2}
$$

$h\nu$ : energia radiației incidente

$L = h\nu_0$ : lucru mecanic de extracție

$\nu_0$ : frecveță de prag
\columnbreak
$$
E c = \frac{mv^2}{2} = e U_S^-
$$

$h = 6,626 \times 10^{-34}\ J s$ : constanta Plank

$|e| = 1,6 \times 10^{-19}\ C$

$c = 3 \times 10^{8}\ m/s$ : viteza luminii
\end{multicols}
\begin{multicols}{2}
$$
p = \frac{h}{\lambda} = \frac{h\nu}{c}
$$

$$
\nu = \frac{c}{\lambda}
$$
\end{multicols}
\section*{Aplicații ale efectului Compton}
\begin{multicols}{2}
	$h\nu = 3,6 \times 10^{-15} J$

	$E c = 8 \times 10^{-20} j$

	\begin{enumerate}[label=\alph*)]
		\item $\nu = ?$
		\item $N = ?$
		\item $L = ?$
		\item $\lambda_0 = ?$
	\end{enumerate}

	$\Delta t = 1s$

	$I = 1mA$

	\columnbreak

	\begin{enumerate}[label=\alph*)]
		\item $\nu = \frac{3,6 \times 10^{-19}}{6,626 \times 10^{-34}}$
		\item $I = \frac{Ne}{\Delta t} \Rightarrow N = \frac{I \Delta t}{e} = \frac{10^{-9}}{1,6 \times 10^{-19}}$
		\item $h\nu = L + E c = 3,6 \times 10^{-19} - 8 \times 10^{-20}$
		\item $L = h\nu_0 = \frac{h c}{\lambda_0}$
	\end{enumerate}
\end{multicols}
\end{document}
